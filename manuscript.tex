% Options for packages loaded elsewhere
\PassOptionsToPackage{unicode}{hyperref}
\PassOptionsToPackage{hyphens}{url}
\PassOptionsToPackage{dvipsnames,svgnames,x11names}{xcolor}
%
\documentclass[
  letterpaper,
  DIV=11,
  numbers=noendperiod]{scrartcl}

\usepackage{amsmath,amssymb}
\usepackage{iftex}
\ifPDFTeX
  \usepackage[T1]{fontenc}
  \usepackage[utf8]{inputenc}
  \usepackage{textcomp} % provide euro and other symbols
\else % if luatex or xetex
  \usepackage{unicode-math}
  \defaultfontfeatures{Scale=MatchLowercase}
  \defaultfontfeatures[\rmfamily]{Ligatures=TeX,Scale=1}
\fi
\usepackage{lmodern}
\ifPDFTeX\else  
    % xetex/luatex font selection
\fi
% Use upquote if available, for straight quotes in verbatim environments
\IfFileExists{upquote.sty}{\usepackage{upquote}}{}
\IfFileExists{microtype.sty}{% use microtype if available
  \usepackage[]{microtype}
  \UseMicrotypeSet[protrusion]{basicmath} % disable protrusion for tt fonts
}{}
\makeatletter
\@ifundefined{KOMAClassName}{% if non-KOMA class
  \IfFileExists{parskip.sty}{%
    \usepackage{parskip}
  }{% else
    \setlength{\parindent}{0pt}
    \setlength{\parskip}{6pt plus 2pt minus 1pt}}
}{% if KOMA class
  \KOMAoptions{parskip=half}}
\makeatother
\usepackage{xcolor}
\setlength{\emergencystretch}{3em} % prevent overfull lines
\setcounter{secnumdepth}{-\maxdimen} % remove section numbering
% Make \paragraph and \subparagraph free-standing
\ifx\paragraph\undefined\else
  \let\oldparagraph\paragraph
  \renewcommand{\paragraph}[1]{\oldparagraph{#1}\mbox{}}
\fi
\ifx\subparagraph\undefined\else
  \let\oldsubparagraph\subparagraph
  \renewcommand{\subparagraph}[1]{\oldsubparagraph{#1}\mbox{}}
\fi


\providecommand{\tightlist}{%
  \setlength{\itemsep}{0pt}\setlength{\parskip}{0pt}}\usepackage{longtable,booktabs,array}
\usepackage{calc} % for calculating minipage widths
% Correct order of tables after \paragraph or \subparagraph
\usepackage{etoolbox}
\makeatletter
\patchcmd\longtable{\par}{\if@noskipsec\mbox{}\fi\par}{}{}
\makeatother
% Allow footnotes in longtable head/foot
\IfFileExists{footnotehyper.sty}{\usepackage{footnotehyper}}{\usepackage{footnote}}
\makesavenoteenv{longtable}
\usepackage{graphicx}
\makeatletter
\def\maxwidth{\ifdim\Gin@nat@width>\linewidth\linewidth\else\Gin@nat@width\fi}
\def\maxheight{\ifdim\Gin@nat@height>\textheight\textheight\else\Gin@nat@height\fi}
\makeatother
% Scale images if necessary, so that they will not overflow the page
% margins by default, and it is still possible to overwrite the defaults
% using explicit options in \includegraphics[width, height, ...]{}
\setkeys{Gin}{width=\maxwidth,height=\maxheight,keepaspectratio}
% Set default figure placement to htbp
\makeatletter
\def\fps@figure{htbp}
\makeatother
\newlength{\cslhangindent}
\setlength{\cslhangindent}{1.5em}
\newlength{\csllabelwidth}
\setlength{\csllabelwidth}{3em}
\newlength{\cslentryspacingunit} % times entry-spacing
\setlength{\cslentryspacingunit}{\parskip}
\newenvironment{CSLReferences}[2] % #1 hanging-ident, #2 entry spacing
 {% don't indent paragraphs
  \setlength{\parindent}{0pt}
  % turn on hanging indent if param 1 is 1
  \ifodd #1
  \let\oldpar\par
  \def\par{\hangindent=\cslhangindent\oldpar}
  \fi
  % set entry spacing
  \setlength{\parskip}{#2\cslentryspacingunit}
 }%
 {}
\usepackage{calc}
\newcommand{\CSLBlock}[1]{#1\hfill\break}
\newcommand{\CSLLeftMargin}[1]{\parbox[t]{\csllabelwidth}{#1}}
\newcommand{\CSLRightInline}[1]{\parbox[t]{\linewidth - \csllabelwidth}{#1}\break}
\newcommand{\CSLIndent}[1]{\hspace{\cslhangindent}#1}

\usepackage{booktabs}
\usepackage{longtable}
\usepackage{array}
\usepackage{multirow}
\usepackage{wrapfig}
\usepackage{float}
\usepackage{colortbl}
\usepackage{pdflscape}
\usepackage{tabu}
\usepackage{threeparttable}
\usepackage{threeparttablex}
\usepackage[normalem]{ulem}
\usepackage{makecell}
\usepackage{xcolor}
\KOMAoption{captions}{tableheading}
\usepackage{float} \floatplacement{figure}{H} \newcommand{\beginsupplement}{\setcounter{table}{0}  \renewcommand{\thetable}{A\arabic{table}} \setcounter{figure}{0} \renewcommand{\thefigure}{A\arabic{figure}}}
\makeatletter
\makeatother
\makeatletter
\makeatother
\makeatletter
\@ifpackageloaded{caption}{}{\usepackage{caption}}
\AtBeginDocument{%
\ifdefined\contentsname
  \renewcommand*\contentsname{Table of contents}
\else
  \newcommand\contentsname{Table of contents}
\fi
\ifdefined\listfigurename
  \renewcommand*\listfigurename{List of Figures}
\else
  \newcommand\listfigurename{List of Figures}
\fi
\ifdefined\listtablename
  \renewcommand*\listtablename{List of Tables}
\else
  \newcommand\listtablename{List of Tables}
\fi
\ifdefined\figurename
  \renewcommand*\figurename{Figure}
\else
  \newcommand\figurename{Figure}
\fi
\ifdefined\tablename
  \renewcommand*\tablename{Table}
\else
  \newcommand\tablename{Table}
\fi
}
\@ifpackageloaded{float}{}{\usepackage{float}}
\floatstyle{ruled}
\@ifundefined{c@chapter}{\newfloat{codelisting}{h}{lop}}{\newfloat{codelisting}{h}{lop}[chapter]}
\floatname{codelisting}{Listing}
\newcommand*\listoflistings{\listof{codelisting}{List of Listings}}
\makeatother
\makeatletter
\@ifpackageloaded{caption}{}{\usepackage{caption}}
\@ifpackageloaded{subcaption}{}{\usepackage{subcaption}}
\makeatother
\makeatletter
\makeatother
\ifLuaTeX
  \usepackage{selnolig}  % disable illegal ligatures
\fi
\IfFileExists{bookmark.sty}{\usepackage{bookmark}}{\usepackage{hyperref}}
\IfFileExists{xurl.sty}{\usepackage{xurl}}{} % add URL line breaks if available
\urlstyle{same} % disable monospaced font for URLs
\hypersetup{
  pdftitle={Binocular integration of chromatic and luminance signals},
  pdfauthor={Daniel H. Baker\^{}\{1,2,*\}; Kirralise J. Hansford\^{}1; Federico G. Segala\^{}1; Annie Y. Morsi\^{}1; Rowan J. Huxley\^{}\{1,3\}; Joel T. Martin\^{}\{1,4\}; Maya Rockman\^{}1; Alex R. Wade\^{}\{1,2\}},
  colorlinks=true,
  linkcolor={blue},
  filecolor={Maroon},
  citecolor={Blue},
  urlcolor={Blue},
  pdfcreator={LaTeX via pandoc}}

\title{Binocular integration of chromatic and luminance signals}
\author{Daniel H. Baker\(^{1,2,*}\) \and Kirralise J.
Hansford\(^1\) \and Federico G. Segala\(^1\) \and Annie Y.
Morsi\(^1\) \and Rowan J. Huxley\(^{1,3}\) \and Joel T.
Martin\(^{1,4}\) \and Maya Rockman\(^1\) \and Alex R. Wade\(^{1,2}\)}
\date{}

\begin{document}
\maketitle
\(^1\)Department of Psychology, University of York, UK

\(^2\)York Biomedical Research Institute, University of York, UK

\(^3\)School of Psychology, University of Nottingham, UK

\(^4\)School of Philosophy, Psychology and Language Sciences, University
of Edinburgh, UK

\(^*\)Corresponding author, email: daniel.baker@york.ac.uk

\hypertarget{abstract}{%
\section{Abstract}\label{abstract}}

Much progress has been made in understanding how the brain combines
signals from the two eyes. However most of this work has involved
achromatic (black and white) stimuli, and it is not clear if the same
processes apply in colour-sensitive pathways. In our first experiment,
we measured contrast discrimination (`dipper') functions for four key
ocular configurations (monocular, binocular, half-binocular and
dichoptic), for achromatic, isoluminant red/green and isoluminant
blue/yellow sine-wave grating stimuli. We find a similar pattern of
results across stimuli, implying equivalently strong interocular
suppression within each pathway. Our second experiment measured
dichoptic masking within and between pathways using the method of
constant stimuli. Masking was strongest within-pathway, and weakest
between chromatic and achromatic mechanisms. Finally, we repeated the
dipper experiment using temporal luminance modulations, which produced
slightly weaker interocular suppression than for spatially modulated
stimuli. We interpret our results in the context of a contemporary
two-stage model of binocular contrast gain control, implemented here
using a hierarchical Bayesian framework. Posterior distributions of the
weight of interocular suppression overlapped with a value of 1 for all
dipper data sets, and the model captured well the pattern of threshold
and slope values from all three experiments.

\textbf{Keywords}: \emph{isoluminant}, \emph{dichoptic}, \emph{binocular
interactions}, \emph{gain control}, \emph{summation},
\emph{suppression}, \emph{masking}, \emph{psychophysics}

\hypertarget{introduction}{%
\section{Introduction}\label{introduction}}

The process by which the brain combines independent inputs is of
fundamental importance for understanding sensory perception. Binocular
vision is a useful test-case for determining the general principles
involved in neural signal combination, as our brains typically combine
the inputs from the left and right eyes to provide binocular single
vision. In recent years our understanding has been facilitated by the
development of binocular gain control models that provide a framework to
interpret empirical data from multiple experimental paradigms and
techniques, including psychophysics (Meese et al., 2006), EEG (Baker \&
Wade, 2017), fMRI (Moradi \& Heeger, 2009) and pupillometry (Segala et
al., 2023). However, the majority of this work has used achromatic
(black and white) stimuli; we know comparatively little about how
chromatic signals are combined binocularly, or about how signals in
different ocular and chromatic channels interact. In this study we use
psychophysical detection and discrimination paradigms to explore
binocular interactions within and between the chromatic and achromatic
pathways.

A useful framework for understanding binocular signal processing is the
two-stage gain control model of binocular combination introduced by
Meese et al. (2006). This model features interocular suppression between
monocular channels, followed by binocular summation. The model accounts
well for the pattern of contrast discrimination (`dipper') functions for
four distinct ocular configurations (see also Georgeson et al., 2016),
illustrated in Figure~\ref{fig-exampledips}. In the monocular condition,
participants must discriminate between stimuli of two contrasts (a
`pedestal', and a `pedestal plus target') that are both presented to one
eye, whilst the other eye views mean luminance. Threshold is defined as
the minimum target contrast required to make this judgement with 75\%
accuracy. The binocular condition is the same, except that the stimuli
are shown to both eyes. In the half-binocular condition the pedestal is
shown to both eyes, but the target increment is shown only to one eye.
Finally, the dichoptic condition involves presenting the pedestal to one
eye, and the target increment to the other eye.

\begin{figure}

{\centering \includegraphics{Figures/exampledips.pdf}

}

\caption{\label{fig-exampledips}Illustration of stimulus conditions
(left) and example dipper functions (right).}

\end{figure}

The detailed pattern of thresholds across these four conditions is
complex, and for achromatic stimuli has several distinctive features
that have been replicated in multiple studies. At low pedestal
contrasts, the binocular condition yields lower thresholds than the
monocular condition; this result is attributed to physiological
binocular summation by neurons responsive to signals from both eyes
(Baker et al., 2018; Campbell \& Green, 1965). However at high pedestal
contrasts the `handle' regions of the dipper functions for these
conditions converge: a consequence of interocular suppression
compensating for the increased excitation during binocular stimulation
(Legge, 1984; Maehara \& Goryo, 2005). The half-binocular condition
avoids confounding the number of eyes seeing the target with the number
of eyes seeing the pedestal (Meese et al., 2006). The pedestal is always
binocular in this condition, whereas the target increment is monocular,
and thresholds are consistently higher than in the binocular condition
across the full range of pedestal contrasts. This demonstrates that
binocular summation occurs across the full contrast range, when the
pedestal ocularity is appropriately controlled. Finally, the dichoptic
condition produces extremely strong masking of the target, such that
when the pedestal is visible, the target must equal or exceed its
contrast in order to be detectable (Baker \& Meese, 2007; Legge, 1979;
Maehara \& Goryo, 2005). The characteristic pattern of dipper functions
is well-described by the gain control model of Meese et al. (2006) for
achromatic stimuli.

At the output of the human retina, cone responses are split into three
distinct pathways. The sum of long- and medium-wavelength cone outputs
(L+M) transmits luminance information, and is likely responsible for the
binocular combination effects previously studied using achromatic
stimuli (see above). The difference of long- and medium-wavelength cone
outputs (L-M) is responsive to chromatic stimuli modulating along a
red/green axis in colour space. Finally the short wavelength cone
outputs (S-(L+M)) code chromatic stimuli modulating along a blue/yellow
axis. There has not yet been a detailed investigation of binocular
contrast discrimination in either of these chromatic pathways, however
there is reason to believe they may differ from the achromatic pathway.
At detection threshold, binocular summation is greater for chromatic
versus achromatic stimuli (Simmons, 2005), implying a more linear
initial stage of processing. For cross-orientation masking, there are
differences in the magnitude of masking between chromatic (red/green)
and achromatic stimuli (Kim et al., 2013; Medina \& Mullen, 2009), as
well as differences in their temporal dynamics (Kim \& Mullen, 2015).
There are also interactions between chromatic and achromatic pathways
both within (Chen et al., 2000) and between (Kingdom \& Libenson, 2015;
Mullen et al., 2014) the eyes, yet these have not been fully explored
for arrangements where the target and mask have the same orientation.
Finally, the neurophysiological underpinnings of colour vision are
distinct from those of the achromatic system. In primary visual cortex
(V1), chromatic signals are processed in `blob' regions that are
revealed by cytochrome oxidase staining (Horton \& Hubel, 1981). The
blob regions appear to be largely monocular (Livingstone \& Hubel,
1984), suggesting that binocular combination for chromatic stimuli might
occur later than for achromatic stimuli, and perhaps be subject to
different constraints.

A further property of `blob' regions is that they are less strongly
orientation-tuned than other regions of V1 (Horton \& Hubel, 1981;
Livingstone \& Hubel, 1984), and biased towards low spatial frequencies
(Edwards et al., 1995; Tootell et al., 1988). Our recent work (Segala et
al., 2023) has investigated binocular combination for flickering discs
of luminance, which are DC-balanced across time (i.e.~the time-averaged
luminance is equal to the background). Steady-state EEG responses from
early visual cortex and psychophysical contrast matching data were both
consistent with weak interocular suppression when using this stimulus
arrangement. This is very different from the `ocularity invariance'
(where binocular and monocular stimuli appear equal) that is
well-established when using DC-balanced periodic stimuli such as
sine-wave gratings, and implies strong interocular suppression (Baker \&
Wade, 2017; Meese et al., 2006; Moradi \& Heeger, 2009). There is
additional psychophysical evidence from matching studies that binocular
combination can be close to linear for luminance increments (Anstis \&
Ho, 1998; Levelt, 1965), particularly against a dark background (Baker
et al., 2012). Both structural and functional data therefore imply that
binocular combination may differ between spatial and temporal contrast.

The main aim of the present study was to characterise binocular signal
combination for chromatic stimuli, and for temporal modulations of
luminance. We also aimed to investigate interocular suppression between
chromatic and achromatic pathways. We therefore preregistered a series
of psychophysical experiments (see: \url{https://osf.io/3vdga/}). In
Experiment 1 we replicate the four key pedestal masking conditions of
Meese et al. (2006) described above for achromatic grating stimuli, and
extend this to both red/green and blue/yellow isoluminant chromatic
stimuli. In Experiment 2 we explore dichoptic masking within and between
these stimuli. Experiment 3 repeats the achromatic condition from the
first experiment, but using a temporally modulated disc rather than
sine-wave gratings. We take a Bayesian approach to data analysis and
modelling; by fitting a hierarchical version of the two-stage gain
control model (Meese et al., 2006) we compare posterior parameter
distributions to understand how model parameters such as the weight of
interocular suppression vary across visual pathways.

\hypertarget{materials-methods}{%
\section{Materials \& Methods}\label{materials-methods}}

\hypertarget{participants}{%
\subsection{Participants}\label{participants}}

All experiments were completed by the first author (DHB) and two
additional participants, who differed for each experiment. Participants
had no known abnormalities of binocular or colour vision. Written
informed consent was obtained before data collection began, and all
procedures were approved by the ethics committee of the Department of
Psychology at the University of York (ID number 2202).

\hypertarget{apparatus-stimuli}{%
\subsection{Apparatus \& stimuli}\label{apparatus-stimuli}}

In Experiments 1 and 2, the stimuli were horizontal sinusoidal gratings
with a spatial frequency of 1c/deg (see examples in
Figure~\ref{fig-examplestim}). The gratings were windowed by a raised
cosine envelope with a diameter of 3 degrees. Spatial phase, relative to
a central fixation cross, was randomised on each trial across the four
cardinal phases. In the achromatic conditions, the sine-wave modulated
all three monitor colour channels equally (red, green and blue). In the
L-M condition, we generated isoluminant stimuli for each participant
(see Procedures) designed to maximise contrast between L and M cones,
whilst keeping S cone activity constant. In the S-(L+M) condition, the
isoluminant stimuli maximised S-cone contrast. Stimuli were converted
from cone space to monitor RGB coordinates using the monitor spectral
readings and the Stockman-Sharpe 2 degree cone fundamentals (Stockman \&
Sharpe, 2000). The stimuli in Experiment 3 were temporal modulations of
luminance applied to a disc made using the same raised cosine envelope
as described above, but with no further spatial modulation. The stimuli
counterphase flickered sinusoidally at 4Hz (see examples in the lower
portion of Figure~\ref{fig-examplestim}). In all experiments, we
displayed a binocular fusion lock, consisting of three concentric rings
of small square elements with random colour. A black central fixation
cross was also displayed throughout.

\begin{figure}

{\centering \includegraphics[width=0.5\textwidth,height=\textheight]{Figures/examplestim.pdf}

}

\caption{\label{fig-examplestim}Example stimuli. Upper row shows grating
stimuli used in Experiments 1 and 2. Lower row shows one cycle of
sinusoidal flicker applied to a uniform disc. Note that the rendering of
all stimuli will depend on the device used to display or print this
image, and so the chromatic stimuli are unlikely to appear isoluminant,
and there may be additional luminance nonlinearities that were not
present in the stimuli displayed during the experiments.}

\end{figure}

All stimuli were presented on an Iiyama VisionMaster Pro 510 CRT
monitor, with a refresh rate of 100Hz, and a resolution of 1024 x 768
pixels. The display was driven by a ViSaGe MkII stimulus generator
(Cambridge Research Systems Ltd., Kent, UK) running in 42-bit colour
mode (14 bits per colour channel). We presented stimuli to the left and
right eyes independently using a four-mirror stereoscope with
front-silvered mirrors. The display was luminance calibrated using a
ColourCal photometer (Cambridge Research Systems), and gamma corrected
by fitting a four-parameter gamma function to the output of each CRT
gun. The maximum luminance was 87 cd/m\(^2\). We also measured the
spectral output of each phosphor using a Jaz spectroradiometer (Ocean
Insight, Florida), and used these measurements to convert between LMS
(cone) space and the monitor RGB coordinates.

For convenience, we express stimulus contrast as a percentage of the
maximum possible contrast that could be displayed on our system. For
achromatic stimuli, the maximum contrast is 1, so this is equivalent to
the standard Michelson contrast expressed as a percentage. For the
isoluminant chromatic stimuli, the maximum displayable L-M (red/green)
cone contrast was 0.1, and the maximum displayable S-(L+M) (blue/yellow)
cone contrast was 0.88. An L-M threshold of 50\% therefore corresponds
to a cone contrast of \(0.5\times0.1 = 0.05\), and an S-(L+M) threshold
of 50\% corresponds to a cone contrast of \(0.5\times0.88 = 0.44\). This
means that the threshold values reported throughout can be converted to
cone contrast by a straightforward multiplicative transform.

\hypertarget{procedure}{%
\subsection{Procedure}\label{procedure}}

All experiments took place in a darkened room. Participants placed their
heads in a chin rest mounted on a height-adjustable table, to which the
stereoscope was also attached. The total optical viewing distance
(including the light path through the mirrors) was 104cm, at which
distance 1 degree of visual angle encompassed 48 pixels on the monitor.

Before beginning primary data collection, each participant in
Experiments 1 and 2 completed an isoluminance adjustment task. Grating
stimuli were presented that counterphase flickered at 5Hz, defined about
either the L-M or S-(L+M) plane in cone space. Participants used a
trackball to dynamically adjust the colour angle of the stimulus to
minimise the percept of flicker. Each participant completed ten such
trials for each colour plane, and the average angle across repetition
was taken as the isoluminant point, and used to generate stimuli for the
main experiment for that participant. Settings were very similar across
participants for the S-(L+M) direction, and somewhat more heterogeneous
for the L-M direction (see Figure~\ref{fig-isofig}).

\begin{figure}

{\centering \includegraphics{Figures/isosettings.pdf}

}

\caption{\label{fig-isofig}Isoluminance settings from all participants
in Experiments 1 and 2. Panel (a) shows red/green and panel (b) shows
blue/yellow settings that were subsequently used to generate stimuli in
the main experiments. Within each panel, solid lines show the mean
settings for each participant, and black curves show the range of
possible stimuli displayed during the adjustment task.}

\end{figure}

In Experiment 1, participants completed a two-interval-forced-choice
(2IFC) contrast discrimination task. Stimuli were presented for 200ms,
with an interstimulus interval of 400ms. Each interval was indicated by
an auditory beep, and participants made their responses using a
two-button trackball. Correct responses were followed by a high pitched
tone, and incorrect responses by a low pitched tone. Each block of the
experiment tested a single pedestal contrast level, and lasted around 12
minutes. On each trial the target contrast level was determined by a
3-down-1-up staircase procedure. There were 8 interleaved staircases in
total; four stimulus arrangements (see Figure~\ref{fig-exampledips})
combined factorially with two target eye assignments. Each pedestal
contrast was repeated 3 times by each participant, and the block order
was randomised. The experiment lasted around 4 hours per participant for
each chromatic condition, and took place over the course of several
weeks. In total, the experiment consisted of 67978 trials (pooled across
participants).

In Experiment 2, participants completed a 2IFC dichoptic masking task.
The stimuli and trial protocol were the same as for Experiment 1, except
that the target contrast was chosen from a set of 10 possible values,
determined in advance based on the data of Experiment 1. There were 12
possible conditions: baseline detection thresholds for achromatic,
red/green and blue/yellow stimuli, and the nine possible factorial
pairings obtained by assigning these conditions to be target and
dichoptic mask stimuli. Mask contrasts were chosen to be approximately
16 times their (monocular) detection threshold, based on the data from
Experiment 1. As a percentage of the maximum displayable contrast, these
were 16\% for achromatic stimuli, and 64\% for both of the chromatic
stimuli. Each block of the experiment tested a single condition, and
consisted of 200 trials. A high contrast example of the target stimulus
was displayed at the foot of the screen throughout, so that there was no
ambiguity about the target identity on a given block. Participants
completed 10 repetitions of each condition (120 blocks of
\textasciitilde6 minutes each), lasting around 12 hours, for a total of
72000 trials (pooled across participants).

In Experiment 3, the achromatic conditions from Experiment 1 were
repeated using a flickering disc stimulus. The stimulus counterphase
flickered at 4Hz, and was presented for 500ms (i.e.~2 full cycles of the
temporal modulation). All other procedures were the same as for
Experiment 1, and the experiment comprised a total of 24610 trials
(pooled across participants).

\hypertarget{data-analysis-and-computational-modelling}{%
\subsection{Data analysis and computational
modelling}\label{data-analysis-and-computational-modelling}}

Psychometric functions from each experiment were fit using
\emph{psignifit} 4 to estimate threshold and slope parameters via a
Bayesian numerical integration method (Schütt et al., 2016). A
cumulative Gaussian was used as the underlying function, with contrast
values expressed in decibel (dB) units (where
\(C_{dB} = 20log_{10}(C_\%)\)). We converted the slope estimates
(\(\sigma\) parameters from the fitted Gaussians) to equivalent Weibull
\(\beta\) values using the approximation \(\beta = 10.3/\sigma\).
Threshold was defined as the target contrast corresponding to an
accuracy of 75\% correct.

The two stage model of Meese et al. (2006) was fit to the threshold data
from Experiments 1 and 3 using a simplex algorithm to minimise the error
between the model and data. The model is defined by a series of
equations:

\begin{equation}
Stage1_L = \frac{C_L^m}{S + C_L + \omega C_R},
\end{equation}

\begin{equation}
Stage1_R = \frac{C_R^m}{S + C_R + \omega C_L},
\end{equation}

\begin{equation}
binsum = Stage1_L + Stage1_R,
\end{equation}

\begin{equation}
Stage2 = \frac{binsum^p}{Z + binsum^q},
\end{equation}

where \(C_L\) and \(C_R\) are the contrasts displayed to the left and
right eyes, and \(m\), \(S\), \(\omega\), \(p\), \(q\) and \(Z\) are
free parameters in the model. A further free parameter, \(k\),
represents additive internal noise, and is used to convert the model
outputs to either d-prime or threshold values (note that in the original
model specification (Meese et al., 2006) this parameter was called
\(\sigma\), but we use the \(k\) symbol here to avoid confusion with the
standard deviation of the cumulative Gaussian used when fitting the
psychometric functions to estimate thresholds). Thresholds are defined
by iteratively adjusting the target contrast until the following
equality is satisfied:

\begin{equation}
Stage2_{target+pedestal} - Stage2_{pedestal} = k,
\end{equation}

and d-prime for a single target level is defined as:

\begin{equation}
d' = \frac{Stage2_{target+pedestal} - Stage2_{pedestal}}{k}.
\end{equation}

We ran the simplex algorithm from 100 random starting vectors for each
data set, and chose the solution for each data set that gave the
smallest root mean squared error (RMSE) between the model and data.

We also implemented a Bayesian hierarchical version of the model using
the Stan language (Carpenter et al., 2017). This used a binomial
generator function to model the proportion correct data at each target
level, and was fit simultaneously to all participants for a given
experiment, but separately for each chromatic condition of Experiment 1,
and the flickering disc data from Experiment 3 (i.e.~four fits in total,
as for the simplex fitting). Prior distributions were Gaussian, with
means determined from published values (see first row of
Table~\ref{tbl-parametertable}). This modelling primarily focuses on
examining posterior parameter distributions, rather than a model
comparison approach. We generated over 1 million posterior samples for
the model, and retained 10\% of them for plotting.

Finally, we adapted the two stage model to include parallel pathways to
process achromatic, red/green and blue/yellow stimuli, that mutually
suppress each other. We added additional suppressive terms at the first
(monocular) stage of the model, for example:

\begin{equation}
ACStage1_L = \frac{AC_L^m}{S + AC_L + \omega_A AC_R + \omega_R RG_R + \omega_B BY_R},
\end{equation}

where \emph{AC} represents the achromatic contrast, \emph{RG} represents
the red/green contrast, \emph{BY} represents the blue/yellow contrast,
and \(\omega_A\), \(\omega_R\) and \(\omega_B\) are the accompanying
weights of interocular suppression. There is an equivalent expression
for the right eye, and for each of the two isoluminant chromatic
pathways. To simplify the model, and avoid free parameters that are
poorly constrained by the data, we fixed several parameters at the
average values from the fits from Experiment 1
(Table~\ref{tbl-parametertable}), such that \(p=7.82\), \(q=6.4\),
\(m=1.21\), \(S=1.03\) and \(k=0.23\). We additionally fixed the
within-mechanism weight of interocular suppression (\(\omega\)) at 1.
This left 9 free parameters: a \(Z\) parameter for each mechanism, and
six cross-mechanism weights of interocular suppression. These parameters
were again estimated within a Bayesian hierarchical framework, using the
trial-wise data from Experiment 2. Note that the model as specified does
not currently include monocular suppression between different pathways,
as we did not collect any data for these conditions. Previous work (Chen
et al., 2000) has measured such interactions, and they could in
principle be incorporated into the denominator of either stage 1 or
stage 2 in the model.

\hypertarget{open-science-practices}{%
\subsection{Open science practices}\label{open-science-practices}}

All experimental code, raw data and analysis scripts are available at:
https://osf.io/3vdga/. The linked GitHub repository also contains a
fully reproducible version of the manuscript. Note that we deviated
slightly from the planned preregistration, in that we did not collect
data for chromatic flickering discs, or for the cross-pathway dichoptic
experiment using disc stimuli. This is because the grating data from
Experiments 1 and 2, and the achromatic disc data from Experiment 3,
were sufficient to address the questions we had hoped to answer from
these experiments.

\hypertarget{results}{%
\section{Results}\label{results}}

\hypertarget{experiment-1}{%
\subsection{Experiment 1}\label{experiment-1}}

Dipper functions from Experiment 1 are displayed in the upper row of
Figure~\ref{fig-dipperfig}. Panel (a) shows the achromatic results,
which replicate the key features from previous work. At detection
threshold, binocular summation was a factor of 1.67 (4.47dB), within the
range (\(\sqrt{2}\) to 2) consistent with previous reports (Baker et
al., 2018). Pedestal masking functions followed the typical `dipper'
shape in all conditions, with a region of facilitation at low pedestal
contrasts, and masking at higher contrasts. The monocular and binocular
dipper handles converge at high contrasts, whereas the half-binocular
thresholds remain above the binocular thresholds across the full range
of pedestal contrasts. The dichoptic condition produced very high
thresholds, with the rising portion of the dipper having a slope around
1 (regression slope of 1.06 in log (dB) units, calculated across the
highest 4 pedestal contrasts).

\begin{figure}

{\centering \includegraphics{Figures/dipperssimplex.pdf}

}

\caption{\label{fig-dipperfig}Dipper functions and psychometric slopes
from Experiment 1, averaged across three participants. Panels (a-c) show
threshold data, and panels (d-f) represent the slope of the psychometric
function expressed in Weibull \(\beta\) units. Error bars give ±1SE
across participants. Note that contrast values are expressed as a
percentage of the maximum displayable contrast (see Procedures for
details). Curves in panels (a-c) show the best fitting models, optimized
using a simplex algorithm, and RMSE values give the root mean squared
errors of the fits.}

\end{figure}

A similar pattern of results was observed for both the red/green and
blue/yellow isoluminant stimuli (see Figure~\ref{fig-dipperfig}b,c).
Summation at threshold was a factor of 1.71 (4.66dB) for the red/green
targets, and a factor of 1.73 (4.77dB) for the blue/yellow targets, and
so was marginally higher than for achromatic stimuli. The general
character of the dipper functions was largely consistent with the
achromatic results, though we observed shallower facilitation and weaker
masking, especially for the blue/yellow stimuli. For example, the
strongest facilitation in the binocular condition for achromatic stimuli
was a factor of 2.63, whereas it reduced to a factor of 2.35 for
red/green stimuli and 1.57 for blue/yellow stimuli. The slope of the
binocular dipper handle was 0.52 for achromatic stimuli, 0.58 for
red/green stimuli, and 0.34 for blue/yellow stimuli. Dichoptic masking
remained as strong for the chromatic conditions as for the achromatic
stimuli (regression slopes of 1.3 for red/green and 1.21 for
blue/yellow). The pattern of results for individual participants was
consistent with the group averages, as shown in
Figure~\ref{fig-individualdippers}.

Following Meese et al. (2006), we also inspected the slope of the
psychometric function for each condition (see
Figure~\ref{fig-dipperfig}d-f). At detection threshold, slopes were
relatively steep for all stimuli, with values around \(\beta=4\). As
pedestal contrasts increased, slopes linearized and reduced to around
\(\beta=1.3\) (Foley \& Legge, 1981; Meese et al., 2006), and remained
shallow at high pedestal contrasts. The exception to this was the
dichoptic condition, where slopes became extremely steep at high
dichoptic mask contrasts, consistent with previous observations (Baker
et al., 2013; Meese et al., 2006). This was clear for all three data
sets, with slope values in the range \(4 < \beta < 8\). However, we
observe that slope estimates are more variable than threshold estimates
(note the large error bars), particularly when using adaptive
staircases, which deploy the majority of trials close to threshold. Our
second experiment therefore investigated the slope of the psychometric
function in more detail for dichoptic masking using the method of
constant stimuli, as well as exploring dichoptic interactions between
chromatic channels.

\hypertarget{experiment-2}{%
\subsection{Experiment 2}\label{experiment-2}}

In Experiment 2 we focussed on the dichoptic condition at a single mask
contrast, and measured full psychometric functions using the method of
constant stimuli for all factorial pairings of target and mask
chromaticity. The pooled results across three participants are shown in
Figure~\ref{fig-MCSfig}a-c, and results for individual participants are
available in Figure~\ref{fig-individualMCS}. All conditions produced
monotonically increasing psychometric functions (panels a-c), but the
extent of masking was highly dependent on the relationship between the
target and mask chromaticity. Figure~\ref{fig-MCSfig}d-f shows a
two-dimensional representation of individual threshold and slope
estimates (points), as well as the posterior density estimates for fits
to the pooled data (ellipses). The results are consistent between
participants, and at the group level, and show that the presence of a
mask has a strong effect on thresholds.

\begin{figure}

{\centering \includegraphics{Figures/MCSdata.pdf}

}

\caption{\label{fig-MCSfig}Summary of data from Experiment 2. Panels
(a-c) show psychometric functions for each condition, pooled across
participants (600 trials per target contrast level). Panels (d-f) show
threshold and slope estimates for individual participants (points) and
the boundary of the posterior density estimates for fits to the pooled
data (ellipses). Panel (g) shows the average threshold elevation factor
for each combination of target and mask stimulus. Panel (h) shows the
geometric mean psychometric slope value for each masking condition in
Weibull \(\beta\) units.}

\end{figure}

Threshold elevation was greatest when the target and mask had the same
chromaticity - notice that the psychometric function is shifted furthest
to the right for the achromatic target with an achromatic mask (white
and black circles in Figure~\ref{fig-MCSfig}a), for the red/green target
with a red/green mask (red and green circles in
Figure~\ref{fig-MCSfig}b), and for the blue/yellow target with a
blue/yellow mask (blue and yellow circles in Figure~\ref{fig-MCSfig}c).
Masking was weakest between achromatic masks/targets and chromatic
masks/targets. Finally there was an intermediate level of masking
between red/green and blue/yellow stimuli. This is summarised in
Figure~\ref{fig-MCSfig}g, which represents threshold elevation for each
combination of target and mask chromaticity. Note that the positive
diagonal exhibits the highest values, and represents threshold elevation
between targets and masks of the same chromaticity.

We also calculated the slope of the psychometric function for each
condition, in equivalent Weibull \(\beta\) units. In the absence of a
mask, the average slope was \(\beta\) = 3.42, which is typical for
contrast detection tasks (Wallis et al., 2013). Slopes became
substantially steeper when the dichoptic mask matched the target in
chromaticity (average \(\beta\) = 5.38). These `super-steep'
psychometric functions for dichoptic pedestal masking have been reported
previously (Baker et al., 2013; Meese et al., 2006), and are observed
for the first time here using chromatic stimuli (see diagonal values in
Figure~\ref{fig-MCSfig}h, and also Figure~\ref{fig-dipperfig}d-f).
However we did not see such markedly steep functions for any of the
cross-chromaticity masking conditions (average \(\beta\) = 2.83 for the
off-diagonal values).

\hypertarget{experiment-3}{%
\subsection{Experiment 3}\label{experiment-3}}

In our final experiment, we again measured dipper functions, but this
time for a disc temporally modulating in luminance. This was motivated
by our recent work (Segala et al., 2023) that appeared to show increased
binocular facilitation and reduced interocular suppression for
flickering disc stimuli (relative to gratings), measured using EEG and a
psychophysical matching paradigm. The pattern of dipper functions for a
4Hz flickering disc (see Figure~\ref{fig-discdata}a) was very similar to
that observed for achromatic gratings (see Figure~\ref{fig-dipperfig}a),
and the binocular summation ratio at threshold was also similar (a
factor of 1.49 for discs, vs 1.67 for gratings). Threshold data for
individual participants are shown in Figure~\ref{fig-individualdiscs}.
We also found a similar pattern of psychometric slope values
(Figure~\ref{fig-discdata}b) as we had for gratings, though we note that
the dichoptic condition did not produce the `super-steep' psychometric
functions we had observed in Experiments 1 \& 2. Nevertheless, the
dichoptic slopes are somewhat above those of the other pedestal
arrangements at high contrasts.

\begin{figure}

{\centering \includegraphics[width=0.8\textwidth,height=\textheight]{Figures/discdata.pdf}

}

\caption{\label{fig-discdata}Thresholds for the flickering disc
experiment. Plotting conventions mirror those in Figure 4.}

\end{figure}

\hypertarget{computational-modelling}{%
\subsection{Computational modelling}\label{computational-modelling}}

\hypertarget{tbl-parametertable}{}
\begin{table}
\caption{\label{tbl-parametertable}Summary of fitted model parameters. The top row gives the best fitting
parameters from the study of Meese et al. (2006). The second section
shows the best fitting parameters from simplex fits to the averaged
thresholds for each experiment. The final rows show the posterior
parameter estimates from the Bayesian model, fitted to each data set
from Experiments 1 and 3. }\tabularnewline

\centering
\begin{tabular}{lrrrrrrrr}
\toprule
\textbf{\em{Model fit}} & \textbf{\em{p}} & \textbf{\em{q}} & \textbf{\em{m}} & \textbf{\em{S}} & \textbf{\em{Z}} & \textbf{\em{$\omega$}} & \textbf{\em{k}} & \textbf{\em{RMSE}}\\
\midrule
Meese et al. (2006) & 7.99 & 6.59 & 1.28 & 0.99 & 0.08 & 1.00 & 0.19 & \\
\midrule
\textbf{Simplex fits} & \textbf{} & \textbf{} & \textbf{} & \textbf{} & \textbf{} & \textbf{} & \textbf{} & \textbf{}\\
Achromatic gratings & 6.51 & 5.10 & 1.31 & 0.83 & 0.16 & 1.00 & 0.21 & 1.59 dB\\
Red/green gratings & 19.42 & 1.10 & 1.03 & 0.37 & 1.49 & 1.00 & 0.29 & 1.43 dB\\
Blue/yellow gratings & 9.47 & 6.21 & 1.14 & 1.89 & 0.26 & 1.00 & 0.34 & 1.38 dB\\
Flickering discs & 6.13 & 4.93 & 1.24 & 0.84 & 0.27 & 0.89 & 0.13 & 1.06 dB\\
\midrule
\textbf{Bayesian model} & \textbf{} & \textbf{} & \textbf{} & \textbf{} & \textbf{} & \textbf{} & \textbf{} & \textbf{}\\
Achromatic gratings & 7.78 & 6.37 & 1.29 & 0.88 & 0.12 & 0.99 & 0.25 & \\
Red/green gratings & 7.88 & 6.46 & 1.21 & 1.10 & 0.46 & 0.95 & 0.21 & \\
Blue/yellow gratings & 7.79 & 6.37 & 1.14 & 1.10 & 0.59 & 1.00 & 0.24 & \\
Flickering discs & 7.72 & 6.32 & 1.26 & 0.83 & 0.22 & 0.90 & 0.23 & \\
\bottomrule
\end{tabular}
\end{table}

For consistency with previous work, we initially performed least-squares
fits of the two-stage model (Meese et al., 2006) for each of the
chromaticity experiments, and the flickering disc experiment (7 free
parameters per fit). The best model fits are shown by the curves in
Figure~\ref{fig-dipperfig} and Figure~\ref{fig-discdata}, and provide an
excellent description of the data, with RMS errors between 1.06 and
1.59dB. Best fitting parameters are shown in the `Simplex fits' section
of Table~\ref{tbl-parametertable}. We note that in previous work, the
weight of interocular suppression (\(\omega\) in the model) is
implicitly fixed at 1. Here we allowed it to vary, but it still received
a value of exactly 1 in all of our grating conditions, and slightly
below 1 (0.89) for our flickering discs. The exponent values (\emph{p}
and \emph{q}) for the red/green gratings are rather different from those
of the other conditions. In previous work (Meese et al., 2006) the
second stage gain control nonlinearity (Equation 4) does typically have
quite substantial exponent values, which balance the relatively mild
nonlinearity at the first stage, and produce a compressive transducer
that results in contrast masking (the handle of the dipper). The high
value of \emph{p} = 19.42 is therefore likely to be compensating for the
low value of \emph{m} = 1.03 at stage one, but may well represent the
combination of several successive stages of nonlinearity, and perhaps
also other phenomena such as uncertainty (Pelli, 1985).

We additionally implemented a hierarchical Bayesian version of the model
in order to estimate full posterior parameter distributions. This model
was fit simultaneously to the full trial-by-trial data from all
participants who participated in a given experiment (i.e.~the model was
fitted separately to each of the four dipper data sets).
Figure~\ref{fig-bayesianmodel}a-d summarises the model behaviour, which
displays the same pattern of dipper functions as we found empirically.
Mean posterior parameter values are given in the lower rows of
Table~\ref{tbl-parametertable}. These correspond quite closely to the
parameters from the simplex fitting and the original Meese et al. (2006)
parameters (reproduced in the first row of the table for reference). The
panels in the right and upper margins of Figure~\ref{fig-bayesianmodel}
show posterior distributions for each model parameter. Note in
particular that the posterior distribution for the weight of interocular
suppression (\(\omega\), top right panel) overlaps 1 for each
experiment, consistent with the strong dichoptic masking observed in the
threshold data. We note that the distribution for \(\omega\) in the
flickering disc condition is somewhat lower than for the other three
data sets. However this difference is not meaningful according to widely
accepted criteria such as comparing the 95\% intervals of the
distribution to a value of \(\omega=1\).

\begin{figure}

{\centering \includegraphics{Figures/stanoutput.pdf}

}

\caption{\label{fig-bayesianmodel}Model predictions (a-d) and posterior
parameters (top and right margin plots) for the hierarchical Bayesian
model. Thick lines in panels (a-d) show curves generated from the mean
posterior parameter estimates. The probability density functions in the
margin plots are peak-normalized, and shown for each of the four data
sets in different colours (see legend in upper left plot). Distributions
were generated from 1000000 samples per data set, using a Markov Chain
Monte Carlo sampling algorithm. Note the logarithmic x-axis for all
posterior plots.}

\end{figure}

Finally we fitted an extended model, that included interocular
suppression between the different pathways, to the data of Experiment 2.
Figure~\ref{fig-MCSmodel} shows the model curves (panels a-c), which
correspond closely to the data in Figure~\ref{fig-MCSfig}. One striking
discrepancy is that the model predicts a region of negative d-prime for
the case where the target and dichoptic mask have the same chromaticity
(see the curved regions below 50\% correct). This happens in the model
because low contrast targets suppress the mask more than they excite the
detecting mechanism, producing a net decrease in response. The feature
(termed a `swan function') was not generally present in our empirical
data, though it can be observed for one participant in
Figure~\ref{fig-individualMCS}b,c.~Our previous work (Baker et al.,
2013) has found evidence for this phenomenon, but it generally requires
very high mask contrasts to be measurable empirically, and there may
also be some individual differences; both factors might explain its
absence here.

Figure~\ref{fig-MCSmodel}d,e shows the model estimates for the weight of
interocular suppression in each combination of target and mask
chromaticity. Although the weights might appear to indicate that
(excluding within-mechanism effects) suppression is strongest between
blue/yellow targets and achromatic masks (a weight of 0.56), this is
somewhat complicated by the differences in mask contrast between
conditions. We selected mask contrasts that were approximately equal
multiples (\(16\times\)) of their own detection thresholds, based on the
data of Experiment 1 (Figure~\ref{fig-dipperfig}a-c). However the
modelling expressed these contrasts as a percentage of the maximum
contrast that it was possible to display on the monitor (see
Procedures). The values in brackets in Figure~\ref{fig-MCSmodel}d
indicate the overall suppressive term on the denominator of the model at
stage 1, calculated by multiplying the mask contrast by the fitted
weight. These values indicate that the strongest suppression is instead
between red/green targets and blue/yellow masks, consistent with the
threshold elevation observed empirically (Figure~\ref{fig-MCSfig}g). In
principle it would be possible to repeat this modelling using either
cone contrast values, or contrast expressed relative to detection
threshold, which would require re-fitting the model.

\begin{figure}

{\centering \includegraphics{Figures/MCSmodel.pdf}

}

\caption{\label{fig-MCSmodel}Summary of the model fit to the data of
Experiment 2. Panels (a-c) show model psychometric functions in the same
format as those in Figure 5. Panel (d) shows the fitted suppressive
weights (\(\omega\) values), with values in brackets indicating the
combined suppressive denominator term calculated by multiplying together
the suppressive weight and the dichoptic mask contrast. Panel (e) shows
the posterior distributions for each weight parameter. The lower right
plot gives the legends for the upper row (top) and the posterior
distributions in panel (e).}

\end{figure}

\hypertarget{discussion}{%
\section{Discussion}\label{discussion}}

Across three psychophysical experiments, we have demonstrated that:

\begin{itemize}
\tightlist
\item
  Binocular combination of isoluminant chromatic stimuli is similar to
  that for achromatic stimuli
\item
  Interocular suppression is strongest within a post-retinal pathway,
  and weakest between achromatic and chromatic pathways
\item
  Binocular combination occurs similarly for spatial and temporal
  luminance modulations
\end{itemize}

We now discuss the relationship to previous work, and consider the
likely physiological substrates of these effects.

\hypertarget{summation-at-threshold}{%
\subsection{Summation at threshold}\label{summation-at-threshold}}

Estimates of the binocular summation ratio at threshold fell within the
range \(\sqrt{2}\) to 2 for all four stimuli tested here. Consistent
with previous reports (Simmons, 2005), summation was slightly higher for
chromatic versus achromatic stimuli, but fell short of perfect linear
summation (a ratio of 2). In our model, summation at threshold is
determined by the exponent at the first gain control stage (the \emph{m}
parameter in Equations 1 \& 2). The summation ratio can be approximated
by \(2^{1/m}\) (see Figure 9 of Baker et al., 2012), such that an
exponent of \(m=1\) produces a ratio of 2, and an exponent of \(m = 2\)
produces a ratio of \(\sqrt{2}\). Both the simplex and Bayesian model
fits (see Table~\ref{tbl-parametertable}) generated parameter estimates
in the range \(1 < m < 1.4\), consistent with the high levels of
summation observed empirically.

\hypertarget{interocular-suppression}{%
\subsection{Interocular suppression}\label{interocular-suppression}}

The weight of interocular suppression is represented by the model
parameter, \(\omega\) (Equations 1 \& 2). Posterior distributions of
this parameter overlapped 1 for all of our dipper function data sets
(see top right panel in Figure~\ref{fig-bayesianmodel}), indicating
strong interocular suppression regardless of post-retinal pathway. The
flickering disc stimuli produced a slightly lower suppression estimate,
with \(\omega \sim 0.9\). This may indicate slightly weaker suppression
between the eyes for temporal modulations, but it is much less extreme
than our recent estimates using EEG and matching paradigms (Segala et
al., 2023). One difference between these studies is the luminance of the
background, which was set to black for the experiments of Segala et al.
(2023), but here was set at the mean luminance. Other studies using
static luminance increments have also reported differences in the
character of binocular combination that are attributable to background
luminance (Baker et al., 2012), so this may explain the differences
between studies. We also note that the weaker interocular suppression
for flickering disc stimuli appears to be largely due to one participant
(see Figure~\ref{fig-individualdiscs}c), so individual differences might
also play a role. Future work should manipulate the background luminance
systematically to better understand how this modulates interocular
suppression.

\hypertarget{psychometric-slopes}{%
\subsection{Psychometric slopes}\label{psychometric-slopes}}

Previous work has demonstrated that the slope of the psychometric
function in 2AFC tasks can distinguish different types of masking,
although it is much less widely reported than threshold measures. In
particular, pedestal masking linearizes the slope (Foley \& Legge, 1981;
Meese et al., 2006), and within-channel dichoptic masking produces very
steep slopes (Baker et al., 2013; Meese et al., 2006). We replicate both
of these effects here, and show that they extend to the isoluminant
chromatic pathways (see Figure~\ref{fig-dipperfig}d-f \&
Figure~\ref{fig-MCSfig}). We additionally show that dichoptic masking
between different pathways does not produce unusually steep slopes (see
Figure~\ref{fig-MCSfig}). It therefore more closely resembles other
types of masking between visual channels, such as cross-orientation
masking (Meese \& Baker, 2009), surround masking (Yu et al., 2002), and
masking from broadband noise (Baker \& Meese, 2012; Lu \& Dosher, 2008),
which also do not impact psychometric slopes.

\hypertarget{physiological-substrates}{%
\subsection{Physiological substrates}\label{physiological-substrates}}

Recent evidence indicates that the physiological substrate of
interocular suppression may be neurons in layer 4 of primary visual
cortex (Dougherty et al., 2019). Most cells in this layer are
monocularly exciteable, in that their responses increase only by
stimulation of their preferred eye. However, simultaneous stimulation of
the non-preferred eye can modulate the response, usually in an
inhibitory fashion, exactly as proposed at stage 1 of the two-stage
model (Equations 1 \& 2). In terms of perception, one consequence of
this early suppression is to achieve `ocularity invariance', whereby the
perceived contrast of a stimulus viewed by one eye is equivalent to that
of the same stimulus viewed by both eyes (Baker et al., 2007). Similar
processes of response invariance have also been reported using fMRI
(Moradi \& Heeger, 2009) and steady-state EEG (Baker \& Wade, 2017).

In V1, chromatic stimuli are processed in `blob' regions that are
largely monocular as they fall within ocular dominance columns
(Livingstone \& Hubel, 1984). Since our psychophysical results indicate
that interocular suppression is equally strong within chromatic and
achromatic pathways, it may be that blob regions include interocular
suppression. Presumably binocular summation of signals occurs at a later
stage of processing for chromatic stimuli, but this does not lead to
different perceptual outcomes. This physiological arrangement might also
be responsible for the pattern of between-pathway interocular
suppression (see Figure~\ref{fig-MCSfig}g). If blobs process signals
from both the L-M and S-(L+M) pathways, their close physical proximity
might explain the stronger suppression between the two chromatic
pathways than between the chromatic and (more distant) achromatic
pathways.

\hypertarget{conclusions}{%
\section{Conclusions}\label{conclusions}}

Here we provide estimates of interocular suppression within and between
the three primary post-retinal visual pathways. These results show that
binocular signal combination is similar within each pathway, but that
interocular suppression is relatively weak between pathways. Our
findings could be applied when building models to predict perception of
binocular images and movies, for example those generated by virtual and
augmented reality systems, or in 3D cinema and television.

\hypertarget{acknowledgements}{%
\section{Acknowledgements}\label{acknowledgements}}

Supported by BBSRC grant BB/V007580/1 awarded to DHB and ARW.

\hypertarget{references}{%
\section{References}\label{references}}

\hypertarget{refs}{}
\begin{CSLReferences}{1}{0}
\leavevmode\vadjust pre{\hypertarget{ref-Anstis1998}{}}%
Anstis, S., \& Ho, A. (1998). Nonlinear combination of luminance
excursions during flicker, simultaneous contrast, afterimages and
binocular fusion. \emph{Vision Res}, \emph{38}(4), 523--539.
\url{https://doi.org/10.1016/s0042-6989(97)00167-3}

\leavevmode\vadjust pre{\hypertarget{ref-Baker2018}{}}%
Baker, D. H., Lygo, F. A., Meese, T. S., \& Georgeson, M. A. (2018).
Binocular summation revisited: Beyond \(\sqrt{2}\). \emph{Psychol Bull},
\emph{144}(11), 1186--1199. \url{https://doi.org/10.1037/bul0000163}

\leavevmode\vadjust pre{\hypertarget{ref-Baker2007b}{}}%
Baker, D. H., \& Meese, T. S. (2007). Binocular contrast interactions:
Dichoptic masking is not a single process. \emph{Vision Res},
\emph{47}(24), 3096--3107.
\url{https://doi.org/10.1016/j.visres.2007.08.013}

\leavevmode\vadjust pre{\hypertarget{ref-Baker2012b}{}}%
Baker, D. H., \& Meese, T. S. (2012). Zero-dimensional noise: The best
mask you never saw. \emph{J Vis}, \emph{12}(10).
\url{https://doi.org/10.1167/12.10.20}

\leavevmode\vadjust pre{\hypertarget{ref-Baker2007}{}}%
Baker, D. H., Meese, T. S., \& Georgeson, M. A. (2007). Binocular
interaction: Contrast matching and contrast discrimination are predicted
by the same model. \emph{Spat Vis}, \emph{20}(5), 397--413.
\url{https://doi.org/10.1163/156856807781503622}

\leavevmode\vadjust pre{\hypertarget{ref-Baker2013}{}}%
Baker, D. H., Meese, T. S., \& Georgeson, M. A. (2013). Paradoxical
psychometric functions ("swan functions") are explained by dilution
masking in four stimulus dimensions. \emph{{iPerception}}, \emph{4}(1),
17--35. \url{https://doi.org/10.1068/i0552}

\leavevmode\vadjust pre{\hypertarget{ref-Baker2017}{}}%
Baker, D. H., \& Wade, A. R. (2017). Evidence for an optimal algorithm
underlying signal combination in human visual cortex. \emph{Cereb
Cortex}, \emph{27}(1), 254--264.
\url{https://doi.org/10.1093/cercor/bhw395}

\leavevmode\vadjust pre{\hypertarget{ref-Baker2012}{}}%
Baker, D. H., Wallis, S. A., Georgeson, M. A., \& Meese, T. S. (2012).
Nonlinearities in the binocular combination of luminance and contrast.
\emph{Vision Res}, \emph{56}, 1--9.
\url{https://doi.org/10.1016/j.visres.2012.01.008}

\leavevmode\vadjust pre{\hypertarget{ref-Campbell1965}{}}%
Campbell, F. W., \& Green, D. G. (1965). Monocular versus binocular
visual acuity. \emph{Nature}, \emph{208}(5006), 191--192.
\url{https://doi.org/10.1038/208191a0}

\leavevmode\vadjust pre{\hypertarget{ref-Carpenter2017}{}}%
Carpenter, B., Gelman, A., Hoffman, M. D., Lee, D., Goodrich, B.,
Betancourt, M., Brubaker, M., Guo, J., Li, P., \& Riddell, A. (2017).
Stan: A probabilistic programming language. \emph{Journal of Statistical
Software}, \emph{76}(1), 1--32.
\url{https://doi.org/10.18637/jss.v076.i01}

\leavevmode\vadjust pre{\hypertarget{ref-Chen2000}{}}%
Chen, C., Foley, J. M., \& Brainard, D. H. (2000). Detection of
chromoluminance patterns on chromoluminance pedestals i: Threshold
measurements. \emph{Vision Res}, \emph{40}(7), 773--788.
\url{https://doi.org/10.1016/s0042-6989(99)00227-8}

\leavevmode\vadjust pre{\hypertarget{ref-Dougherty2019}{}}%
Dougherty, K., Cox, M. A., Westerberg, J. A., \& Maier, A. (2019).
Binocular modulation of monocular V1 neurons. \emph{Curr Biol},
\emph{29}(3), 381--391.e4.
\url{https://doi.org/10.1016/j.cub.2018.12.004}

\leavevmode\vadjust pre{\hypertarget{ref-Edwards1995}{}}%
Edwards, D. P., Purpura, K. P., \& Kaplan, E. (1995). Contrast
sensitivity and spatial frequency response of primate cortical neurons
in and around the cytochrome oxidase blobs. \emph{Vision Res},
\emph{35}(11), 1501--1523.
\url{https://doi.org/10.1016/0042-6989(94)00253-i}

\leavevmode\vadjust pre{\hypertarget{ref-Foley1981}{}}%
Foley, J. M., \& Legge, G. E. (1981). Contrast detection and
near-threshold discrimination in human vision. \emph{Vision Res},
\emph{21}(7), 1041--1053.
\url{https://doi.org/10.1016/0042-6989(81)90009-2}

\leavevmode\vadjust pre{\hypertarget{ref-Georgeson2016}{}}%
Georgeson, M. A., Wallis, S. A., Meese, T. S., \& Baker, D. H. (2016).
Contrast and lustre: A model that accounts for eleven different forms of
contrast discrimination in binocular vision. \emph{Vision Res},
\emph{129}, 98--118. \url{https://doi.org/10.1016/j.visres.2016.08.001}

\leavevmode\vadjust pre{\hypertarget{ref-Horton1981}{}}%
Horton, J. C., \& Hubel, D. H. (1981). Regular patchy distribution of
cytochrome oxidase staining in primary visual cortex of macaque monkey.
\emph{Nature}, \emph{292}(5825), 762--764.
\url{https://doi.org/10.1038/292762a0}

\leavevmode\vadjust pre{\hypertarget{ref-Kim2013}{}}%
Kim, Y. J., Gheiratmand, M., \& Mullen, K. T. (2013). Cross-orientation
masking in human color vision: Application of a two-stage model to
assess dichoptic and monocular sources of suppression. \emph{J Vis},
\emph{13}(6), 15. \url{https://doi.org/10.1167/13.6.15}

\leavevmode\vadjust pre{\hypertarget{ref-Kim2015}{}}%
Kim, Y. J., \& Mullen, K. T. (2015). The dynamics of cross-orientation
masking at monocular and interocular sites. \emph{Vision Res},
\emph{116}(Pt A), 80--91.
\url{https://doi.org/10.1016/j.visres.2015.09.008}

\leavevmode\vadjust pre{\hypertarget{ref-Kingdom2015}{}}%
Kingdom, F. A. A., \& Libenson, L. (2015). Dichoptic color saturation
mixture: Binocular luminance contrast promotes perceptual averaging.
\emph{J Vis}, \emph{15}(5), 2. \url{https://doi.org/10.1167/15.5.2}

\leavevmode\vadjust pre{\hypertarget{ref-Legge1979}{}}%
Legge, G. E. (1979). Spatial frequency masking in human vision:
Binocular interactions. \emph{J Opt Soc Am}, \emph{69}(6), 838--847.
\url{https://doi.org/10.1364/josa.69.000838}

\leavevmode\vadjust pre{\hypertarget{ref-Legge1984}{}}%
Legge, G. E. (1984). Binocular contrast summation--II. Quadratic
summation. \emph{Vision Res}, \emph{24}(4), 385--394.
\url{https://doi.org/10.1016/0042-6989(84)90064-6}

\leavevmode\vadjust pre{\hypertarget{ref-Levelt1965}{}}%
Levelt, W. J. (1965). Binocular brightness averaging and contour
information. \emph{Br J Psychol}, \emph{56}, 1--13.
\url{https://doi.org/10.1111/j.2044-8295.1965.tb00939.x}

\leavevmode\vadjust pre{\hypertarget{ref-Livingstone1984}{}}%
Livingstone, M. S., \& Hubel, D. H. (1984). Anatomy and physiology of a
color system in the primate visual cortex. \emph{J Neurosci},
\emph{4}(1), 309--356.
\url{https://doi.org/10.1523/JNEUROSCI.04-01-00309.1984}

\leavevmode\vadjust pre{\hypertarget{ref-Lu2008}{}}%
Lu, Z.-L., \& Dosher, B. A. (2008). Characterizing observers using
external noise and observer models: Assessing internal representations
with external noise. \emph{Psychol Rev}, \emph{115}(1), 44--82.
\url{https://doi.org/10.1037/0033-295X.115.1.44}

\leavevmode\vadjust pre{\hypertarget{ref-Maehara2005}{}}%
Maehara, G., \& Goryo, K. (2005). Binocular, monocular and dichoptic
pattern masking. \emph{Optical Review}, \emph{12}(2), 76--82.
\url{https://doi.org/10.1007/PL00021542}

\leavevmode\vadjust pre{\hypertarget{ref-Medina2009}{}}%
Medina, J. M., \& Mullen, K. T. (2009). Cross-orientation masking in
human color vision. \emph{J Vis}, \emph{9}(3), 20.1--16.
\url{https://doi.org/10.1167/9.3.20}

\leavevmode\vadjust pre{\hypertarget{ref-Meese2009}{}}%
Meese, T. S., \& Baker, D. H. (2009). Cross-orientation masking is speed
invariant between ocular pathways but speed dependent within them.
\emph{J Vis}, \emph{9}(5), 2.1--15. \url{https://doi.org/10.1167/9.5.2}

\leavevmode\vadjust pre{\hypertarget{ref-Meese2006}{}}%
Meese, T. S., Georgeson, M. A., \& Baker, D. H. (2006). Binocular
contrast vision at and above threshold. \emph{J Vis}, \emph{6}(11),
1224--1243. \url{https://doi.org/10.1167/6.11.7}

\leavevmode\vadjust pre{\hypertarget{ref-Moradi2009}{}}%
Moradi, F., \& Heeger, D. J. (2009). Inter-ocular contrast normalization
in human visual cortex. \emph{J Vis}, \emph{9}(3), 13.1--22.
\url{https://doi.org/10.1167/9.3.13}

\leavevmode\vadjust pre{\hypertarget{ref-Mullen2014}{}}%
Mullen, K. T., Kim, Y. J., \& Gheiratmand, M. (2014). Contrast
normalization in colour vision: The effect of luminance contrast on
colour contrast detection. \emph{Sci Rep}, \emph{4}, 7350.
\url{https://doi.org/10.1038/srep07350}

\leavevmode\vadjust pre{\hypertarget{ref-Pelli1985}{}}%
Pelli, D. G. (1985). Uncertainty explains many aspects of visual
contrast detection and discrimination. \emph{J Opt Soc Am A},
\emph{2}(9), 1508--1532. \url{https://doi.org/10.1364/josaa.2.001508}

\leavevmode\vadjust pre{\hypertarget{ref-Schutt2016}{}}%
Schütt, H. H., Harmeling, S., Macke, J. H., \& Wichmann, F. A. (2016).
Painfree and accurate bayesian estimation of psychometric functions for
(potentially) overdispersed data. \emph{Vision Res}, \emph{122},
105--123. \url{https://doi.org/10.1016/j.visres.2016.02.002}

\leavevmode\vadjust pre{\hypertarget{ref-Segala2023}{}}%
Segala, F. G., Bruno, A., Martin, J. T., Aung, M. T., Wade, A. R., \&
Baker, D. H. (2023). Different rules for binocular combination of
luminance flicker in cortical and subcortical pathways. \emph{eLife},
\emph{12}, RP87048. \url{https://doi.org/10.7554/eLife.87048}

\leavevmode\vadjust pre{\hypertarget{ref-Simmons2005}{}}%
Simmons, D. R. (2005). The binocular combination of chromatic contrast.
\emph{Perception}, \emph{34}(8), 1035--1042.
\url{https://doi.org/10.1068/p5279}

\leavevmode\vadjust pre{\hypertarget{ref-Stockman2000}{}}%
Stockman, A., \& Sharpe, L. T. (2000). The spectral sensitivities of the
middle- and long-wavelength-sensitive cones derived from measurements in
observers of known genotype. \emph{Vision Res}, \emph{40}(13),
1711--1737. \url{https://doi.org/10.1016/s0042-6989(00)00021-3}

\leavevmode\vadjust pre{\hypertarget{ref-Tootell1988}{}}%
Tootell, R. B., Silverman, M. S., Hamilton, S. L., De Valois, R. L., \&
Switkes, E. (1988). Functional anatomy of macaque striate cortex. III.
color. \emph{J Neurosci}, \emph{8}(5), 1569--1593.
\url{https://doi.org/10.1523/JNEUROSCI.08-05-01569.1988}

\leavevmode\vadjust pre{\hypertarget{ref-Wallis2013}{}}%
Wallis, S. A., Baker, D. H., Meese, T. S., \& Georgeson, M. A. (2013).
The slope of the psychometric function and non-stationarity of
thresholds in spatiotemporal contrast vision. \emph{Vision Res},
\emph{76}, 1--10. \url{https://doi.org/10.1016/j.visres.2012.09.019}

\leavevmode\vadjust pre{\hypertarget{ref-Yu2002}{}}%
Yu, C., Klein, S. A., \& Levi, D. M. (2002). Facilitation of contrast
detection by cross-oriented surround stimuli and its psychophysical
mechanisms. \emph{J Vis}, \emph{2}(3), 243--255.
\url{https://doi.org/10.1167/2.3.4}

\end{CSLReferences}

\hypertarget{appendices}{%
\section{Appendices}\label{appendices}}

\beginsupplement

\begin{figure}

{\centering \includegraphics{Figures/individualdippers.pdf}

}

\caption{\label{fig-individualdippers}Individual participant data from
Experiment 1.}

\end{figure}

\begin{figure}

{\centering \includegraphics{Figures/individualMCS.pdf}

}

\caption{\label{fig-individualMCS}Individual participant data from
Experiment 2.}

\end{figure}

\begin{figure}

{\centering \includegraphics{Figures/individualdiscs.pdf}

}

\caption{\label{fig-individualdiscs}Individual participant data from
Experiment 3.}

\end{figure}



\end{document}
